\chapter{Introduction}
\label{chapter_1}

\begin{abstract}
This is the abstract of the introduction
\end{abstract}

\newpage

\section{Light Microscopy}
\dropcap{M}{icroscopes} have become an indispensable tool in all research fields
in which there is a need to look at small features. The first microscopes
developed by Antoni Van Leeuwenhoek in the XVII century were aimed at studying
fabrics; it didn't take long however to discover that nature was hiding amazing
elements beyond what the bare human eye could see. The first microscopes focused
into developing better lenses and clever illumination schemes. 

With the development of the ondulatory theory of light a fundamental limitation
for optical microscopes appeared: the diffraction limit. Abbe realized that no
matter how good a lens is, there will always be a limit to how much it is
possible to focus light. This limit is determined mainly by the wavelength of
the employed light beam and by the maximum angle the lens is able to focus. 

The diffraction limit puts a restriction to the size of the structures that can
be resolved under an optical microscope. The use of shorter wavelengths, as
X-Rays opened the possibility to study much smaller structures. However this was
at an expense of observing very well defined periodic structures, such as
crystals. Soft matter samples such as cells would therefore be out of the scope
of these techniques. 

It was at the end of the XX century however that a major breakthrough occurred
in the field of optics: the detection of a single-molecule fluorescence by M.
Orrit and J. Bernard. Single-molecules opened the door to studying materials 
with unprecedented spatial resolution, but also to determine properties that
would have been hidden by bulk broadening. The first studies were done at low
temperature (few Kelvins) and allowed to determine properties not only of the
fluorescent molecules but also of the hosting matrices, mainly polymers and
crystals. 

With a growing interest in the field, a big effort was placed in allowing the
detection of single-molecules at room temperature. This led to the development
of new organic dyes and to establish single-molecule fluorescence microscopy as
one of the cornerstones of many research fields. Localization of single
fluorophores led to the development of what is now known as super resolution
microscopy. By carefully determining the centroid of the emission pattern, it is
possible to determine the center a molecule with higher accuracy than what the
diffraction limit would allow. 

Molecules however show blinking and bleaching. At room temperature it is
impossible to prevent fluorophores from going to dark states, meaning that their
fluorescence signal will disappear either for a short period of time or for the
remaining time of the experiment. This puts a hard limit to the experiments that
can be performed employing single-molecules, since they cannot be observed for
extended periods of time. Tracking is limited to few seconds, imaging is limited
to few frames or to cleverly engineered illumination strategies. 

As single-molecule detection allowed to bridge the length mismatch between
visible light and biologically relevant scales, new agents that can fill the gap
between biologically relevant time scales and fluorophores' observation times
are of utmost importance. In this direction different approaches were taken,
including employing scattering instead of fluorescence, the use of semiconductor
quantum dots and of metallic nanoparticles. The latter are the focus of this
thesis and of the next few sections. 

\section{Gold Nanoparticles}
Metallic nanoparticles have been subject of studies for a long time. In a
fortuitous way Romans managed to generate red-coloured glass by dispersing gold
nanospheres into their glass mixing strategies; the beautiful Lycurgus cup is
the only surviving example of such technique, together with some other glass
fragments of the time. The nanoparticles in the glass preserved their optical
properties for centuries, however the explanation of the phenomenon came several
centuries later.

Gustav Mie in $1908$ calculated the scattering of a plane wave incident on
spherical particles. It relies on fully solving Maxwell equations and nowadays
it is simply known as Mie scattering. In the original paper it is possible to
observe the resonance of gold nanoparticles at around $550\nm$; both calculation
and measurements show a peak in the scattering efficiency at those wavelengths.
Because of the weaker interaction with light of longer wavelengths, the reddish
color of colloidal gold nanoparticles can be explained.

Metals however show another interesting property that is given by the
oscillation of conduction electrons and is known as plasmon. For particles much
smaller than the incident wavelength, a simplification of the Mie formalism can
be made by considering only the first order. In this case the polarizability of
a nanosphere is given by
\begin{equation}label{eqn:polarizability}
	\alpha_{\textrm{sphere}} =
	3\epsilon_0V\frac{\epsilon(\omega)-\epsilon_m}{\epsilon(\omega)+2\epsilon_m}
\end{equation}
where $\epsilon_0$ is the permittivity of vacuum, $\epsilon(\omega)$ is the
permittivity of the metal as function of incoming excitation frequency $\omega$
and $\epsilon_m$ is the permittivity of the surrounding medium. The absorption
cross section can thus be calculated as
$\sigma_\textrm{abs}=k\textrm{Im}(\alpha)$ and the scattering as
$\sigma_\textrm{scatt}=k^4|\alpha|^2/(6\pi)$.

From equation \ref{eqn:polarizability} it is possible to see that a resonance
will appear when $\textrm{Re}(\epsilon(\omega)) = -2\epsilon_\textrm{m}$. It is
important to note that the resonance is therefore dependent not only on the
particle's material properties but also on the surrounding medium's optical
constants. In the case of elongated nanoparticles, some correction factors can
be introduced to the polarizability. However several computer packages exist to
calculate with a great precision absorption and extinction cross sections of
arbitrary geometries. 

A standard procedure to obtain gold nanoparticles is through wet chemical
methods. Even in the best of cases there will be a dispersion in shapes that
will give rise to inhomogeneous broadening. Some interesting properties will be
concealed in bulk measurements, therefore making single-particle experiments of
great importance. 

Light emission from gold and copper was observed by
Mooradian\cite{Mooradian1969} in $1969$. In this work, electron and holes in the
metal were excited with visible light and the emission was observed at longer
wavelengths. Strikingly, the emission quantum yield (i.e. the number of
emitted photons per absorbed photon) was in the order of $10^{-10}$. In
subsequent years some studies showed that this low number could be increased
with the presence of sharp edges or tips, but still it would be much lower that
what is observed for an organic dye, in the order of few percent at least. 



