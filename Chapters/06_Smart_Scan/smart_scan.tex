\chapter[Smart Scan]{Smart Scan: A Python program to control scanning
microscopes}
\label{ch:Damping}

\begin{abstract}
This is the abstract for the computer program
\end{abstract}

\newpage

\section{Scheme of the chapter}
* Motivations (why another program)
  * Research greatly depends on computer power
  * Computers have to be considered tools
  * Reinventing the wheel
  * Commercial systems are normally closed
  * General programming languages have a steep learning curve
  * Our niche: Flexible but low level
* MVC Design pattern
  * What is a design pattern for a computer program
  * Reusability of code
  * MVC in the lab
  * The ADwin box
* The program itself
  * Going from low-level adq class
  * To high level keep track of a particle
  * Building a GUI
  * Interfacing with serial devices
  * Interfacing with an Arduino
  * Interfacing through the network
  * Building an API
  * Starting the program from a web page
* Next steps
  * New applications
  * Building a company around the program


\section{Why another computer package}
Computers became an indispensable tool in almost every aspect of life. Research
labs are not an exception; even the simplest measurement with a digital scale
depends on electronics. When dealing with more complex setups and experiments
also the complexity of the acquisition devices grows. Nowadays digitizing an
analog signal or generating an analog output from a digital controller is a
standard procedure that doesn't require really expensive equipment nor a
highly specialized training. 

In research labs however, there is a big gap between the experiments that have
to be performed and the use of computers to simplify those tasks. Good
researches may not have advance training in computer science and therefore the
computer may look as an obstacle between their research and themselves. In
laboratories that don't have their focus in advanced instrumentation a lot of
repetitive, non creative tasks are performed by people instead of being
automatized. Computers have to be thought as tools that can greatly simplify 
researchers work, and this chapter focuses on how to achieve it. 

The risk when implementing a new solution is to reinvent the wheel. Sometimes
building a program from scratch gives insight into all the steps involved into
making the setup work, but how deep one should dig depends on the focus of the
research. When the field is not instrumentation in itself, most likely there
have been groups already working in similar techniques. For example performing a
confocal scan in an optical microscope should be a routine work. However the
development of a computer program to control all the involved pieces, namely a
piezo stage and a detector, can take longer than setting up and aligning all the
optics. 

It would be expected that computer packages already exist to solve this common
problem. However the majority of them can be installed only into very specific
hardware configurations. The diversity that exists between setups is so large
that being able to cover all possible configurations is not a realistic task.
Therefore being able to build a computer program on top of accumulated knowledge
can be of great use in a variety of research fields, including but not limited
to optical setups. This lower level program should be therefore transparent to
the user; in case of need it should provide easy ways of customization but it
should also seamlessly interface with higher level needs. 

One of the key requisites of computer packages aimed to scientific research is
that they should be open source for anybody to inspect and improve. Even if
the software is propietary its source code can be released in order to provide
all the flexibility researchers need. Moreover computer tools used for
acquiring, simulating or analyzing data could in principle be the object of a
peer review process before accepting a publication. 

In the scientific community Python has gained a lot of attention; on one hand it
is an open source language, that can be installed in any operating system
without the need of expensive licenses. On the other hand it posses extensions
that are ideal for scientific data analysis, simulations and for building
complex applications. Python is an interpreted language, meaning that it does
not need to be compiled prior to execution. This shortens the development time
of new programs, while maintains a high versatility. Moreover Python counts
with an active community that makes relatively easy to find answers to common
questions online.

Learning a computer language can take more time than available. Experimentalists
can find themselves in the crossroad of choosing if investing time in learning
to code or to focus into the experiments themselves. Learning to code may a be a
good long term strategy, but probably the impact in the research is too long
term for the current scientific output needs. Learning from examples however can
shorten the learning curve significantly. A computer package that can already
perform experiments similar to the ones planned can be easily adapted. In this
thesis for example, the code used in chapter \ref{ch:KCN} for refocusing onto
the particles was the same employed in chapters \ref{ch:AntiStokes} and
\ref{ch:Damping} for tracking the displacement of a reference particle with
increasing temperature. 

The package that we have developed follows all the guidelines stated above. It
is open source, built on an open source language. It provides a low level layer
responsible of interacting with the devices and several upper level examples on
how to operate with it. Lower level features can be added to the base code, but
at the same time new experiments can be planned without having to worry about
how to exactly interface with individual devices. This puts our computer package
into a specific niche that is not covered by others: it provides a low level
layer and at a same time maintains a great extent of flexibility. 

\section{The MVC design pattern}
The model-view-controller (MVC) design pattern is a standard strategy at the
time of planning code for ensuring its reusability. Each layer will deal with
different parts of the process, meaning that the model will be responsible for
managing the data, analyzing it. The data is then passed to the view layer that
is responsible for presenting it to the user with its own set of rules. The user
can send instructions through the controller to update the model, for instance
by acquiring more data. 

The MVC design patter allows to split the different programming tasks in such a
way that different programmers can focus into specific tasks. It also allows to
interchange different pieces of code, for example it is possible to use the same
controller but to change the view layer. In the case of a microscope, same
controller can have the ability to perform scans, i.e. acquiring a signal while
varying one or more outputs, and to acquire timetraces. However the way data is
presented to the user is significatively different: it can range from a 3D model
to a 1D plot. The model layer is responsible for interpreting the acquired data,
for example by converting it from a digital value to a physical quantity. 

In the lab the MVC design pattern arises naturally when thinking in the
different steps involved in an experiment. First there is the experimentalist,
or the user who needs to interact with the computer to instruct what measurement
to perform. This triggers a device that can digitize a signal, output a
voltage, etc. In turn the acquired data is transformed into something that can
be easily interpreted by the user, such as a value on the screen or a plot. In
this simple description the three layers of the MVC pattern have a clear
function. 

The reusability of code is especially useful for scientific applications. It
does not only shorten the development time it also allows less experienced users
to build complex software. The main characteristic of reusable code besides
the design pattern is its documentation. In such regard Python allows to build
documentation pages based on the information supplied in the code.
Automatically generated manual pages, even if not pleasant to read, allow to
search for features, avoid errors and bugs and to continue building on top of
others experience. 

One of the key elements in the acquisition scheme is the hardware able to
digitize signals and to generate outputs. In our case we employed an ADwin box,
capable of handling several inputs and outputs both analog and digital. The
ADwin box can run dedicated tasks with high time accuracy in principle without
interacting with a PC. The tasks are written in a specific language similar to
Basic, compiled and loaded into the box. The interaction with the computer is
achieved through specific variables that can be read and written both from the
ADwin box and from the PC.

\section{SMART Scan}
The name Smart Scan has its origin in the original purpose of the code written.
SM stands for single-molecule and AR for Automatic Refocusing. The initial idea
was to develop a simple program that could automatically refocus a confocal
microscope on a single gold nanorod after it drifted away some micrometers. 
