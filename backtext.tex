Gold nanorods are excellent candidates to act as nanoprobes because they are
reasonably bright upon excitation with a monochromatic source. We study the
luminescence properties of gold nanorods and show that they are efficient
emitters at wavelengths shorter than the excitation wavelength. We show that
this anti-Stokes emission can be used for imaging under high-background
conditions such as stained cells. We also show that the spectrum of the
anti-Stokes emission can be used to measure the temperature of the nanoparticles
upon excitation with a laser. This technique will open new and exciting
possibilities in the field of photothermal therapy.
