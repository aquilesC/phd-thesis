\documentclass{dissertation}

%% Turn off page numbering for the propositions and make the margins on both
%% sides equal and symmetrical.
\geometry{twoside=false}
\pagestyle{empty}

\begin{document}

%% Specify the title and author of the thesis. This information will be used on
%% both the English and Dutch side and in the metadata of the final PDF.
\title[Applications to Imaging and Temperature Sensing]{Gold Nanorod
Photoluminescence}
\author{Aquiles}{Carattino}

\begin{center}

{\Large\titlefont\bfseries Stellingen}

\bigskip

behorende bij het proefschrift

\bigskip

%% Print the title.
{\makeatletter
\titlestyle\bfseries\large\@title
\makeatother}

%% Print the optional subtitle.
{\makeatletter
\ifx\@subtitle\undefined\else
    \titlefont\titleshape\@subtitle
\fi
\makeatother}

\bigskip

\end{center}

\bigskip
\bigskip

\begin{enumerate}

\item Heterogeneity among gold nanorods goes beyond what experimental accuracy
can resolve. This is a major drawback that has to be addressed in the future. \\
\textit{Chapters 2 and 4 of this thesis} 

\item It is not an \textit{a priori} requirement to fully understand a
phenomenon in order to successfully exploit it.
\\
\textit{Chapter 3 of this thesis} 

\item It is important to study nanoparticles with different geometries; however
this type of work tends to be of a lower impact and less motivating that
focusing into a new or not yet explained phenomenon.
\\
\textit{Chapter 4 and 5 of this thesis}

\item A complete model to explain the luminescence of single gold nanoparticles
is missing; more accurate temperature measurements and planning future
experiments will depend on having such a model.
\\
\textit{Chapter 4 of this thesis}

\item Measuring temperature with surface enhanced Raman spectroscopy
(SERS), only allows to determine temperature in the hotspot. \\
\textit{Pozzi et al., J. Phys. Chem. C 119, 21116-21124 (2015).}

\item The future of plasmonic enhanced fluorescence correlation spectroscopy
(FCS) may lie in cleverly designed nanoparticles. \\
\textit{Langguth et al., Opt. Express 22, 15397 (2014).}

\item Thermometry in living cells needs very careful data analysis and
interpretation. Sloppy raw data processing may lead to artifacts and false
conclusions.
\\
\textit{Baffou et al., Nat. Methods 11, 899–901 (2014).}

\item Superlocalization requires more than fitting bright pixels by a
Gaussian. When dealing with new sources of light it is important to verify
the hypothesis that lead to accuracies beyond the diffraction limit.\\
\textit{Titus et al., ACS Nano 7, 6258–6267 (2013).}

\item Gender inequality has to be addressed in a definitive way. The scientific
\textit{system} is no example to follow by the rest of society.

\item Failed research is much riskier for a PhD candidate than for someone with
a permanent position; risk management should always be a priority.

\item Researchers should always keep an eye on societal issues. Research and its
applications should not be confined to academic environments.

\item Reverting the paper-frenzy of the last several years is a collective
effort that has to be agreed by different actors, from researches to governing
bodies, and specially by \textit{us}.

\end{enumerate}

\bigskip
\bigskip

%% Apart from the name and title of the supervisor, the following text is
%% dictated by the promotieregelement.
\begin{center}
Aquiles Carattino \\
Leiden, February 30, 2017
\end{center}

\end{document}

