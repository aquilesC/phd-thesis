\chapter*{Preface}
\setheader{Preface}

Behind every thesis lies a story that involves much more people than the
mere author whose name is on the cover. This book is the conclusion of four years of
work in the MoNOS group at the physics institute in Leiden, where I have focused
on the study of single gold nanorod luminescence and its possible applications. 

The project was framed within a larger collaboration with three more groups from
biophysics, biology and chemistry. The aim of the project was utilizing single
gold nanorods as labels in the nucleus of cells, focusing into the study of the
glucocorticoid receptor. My part in the collaboration was the understanding of
the mechanisms that give rise to the luminescence of single gold nanorods,
crucial for the imaging and tracking in living cells. Even if my work was not
biophysical, I always kept an eye into the biological applications of my
research.

A Friday afternoon idea, triggered by postdoc Saumyakhanti Khatua evolved into
what now is chapter \ref{ch:KCN} of this booklet. He asked me if it was possible
to monitor the etching process of gold by cyanide ions in single gold
nanoparticles. The first trial showed already something interesting: single
particles on glass were behaving completely different than in bulk suspension.
More importantly, to answer the question it was clear that we needed better
software to control the setup.

Ferry Kruidenberg, a bachelor student at the time, joined the group to help
develop the software even further. He designed the core layout of the program
and the first graphical user interface. Together we learned about version
control, instrumentation and programming patterns. Simultaneously, a master
student, Irina Komen, joined the group to start working on an optical tweezer.
Together we managed to obtain photothermal signals from single nanoparticles
that were trapped in water and glycerol. However, the final objective was to
study the fluorescence enhancement of a dye in the vicinity of the rod while
away from any other surface, thus preventing sticking of the molecules to the
coverslip.

Even if the results on the optical tweezer didn't fit into this thesis, while
characterizing the emission from different nanoparticles an interesting
phenomenon appeared: emission at higher energies than the excitation energy, the
anti-Stokes emission. This emission proved to be reasonably efficient, sometimes
even comparable to the Stokes-shifted counterpart. The detection of anti-Stokes
luminescence was consistent between different samples and under different
conditions.

Anti-Stokes luminescence opened the door to two different approaches. Firstly it
was possible to exploit the emission at shorter wavelengths to suppress the
background when imaging under biological conditions. Cells are known to
fluoresce under laser irradiation and therefore dim emitters such as small nanoparticles
are hard to distinguish from the background. After discussing with Veer Keizer,
a PhD candidate in biology belonging to the same project, we embarked into the
exploitation of the anti-Stokes emission for imaging. These ideas led to
Chapter \ref{ch:Imaging} and its publication in the Biophysical Journal. It was
very well received by the reviewers, one of them qualified the findings as
a ``very important breakthrough''.

However, there was more in the anti-Stokes emission than solely the application
to imaging. If the emission depends on temperature, it can be used as a
nano-thermometer. Photothermal therapy is a fertile subject that relies on
locally increasing the temperature to kill specific cells. This is achieved by
shining a laser onto gold nanoparticles inside or in the vicinity of those
cells. However there are so far no ways of controlling the temperature of the
particles. Studying the anti-Stokes emission can be a solution to a long
standing problem in the medical and biological sciences.

To prove the usefulness of the method, the measurements were performed in a
temperature variable flow cell. An air spaced objective was needed to avoid
altering the temperature of the observed area, which in turn lowered the
collection efficiency. At higher temperatures (around $60\degree$) the setup
drifts several micrometers and therefore an accurate control of temperature and
a proper tracking of the particles was needed. 

Chapter \ref{ch:AntiStokes} shows that it is indeed possible to determine the
absolute temperature of single nanoparticles just by measuring their anti-Stokes
spectrum. The method does not require any form of \textit{ad-hoc} calibration
and can be easily implemented in any confocal microscope coupled with a
spectrometer. These findings can have a major impact on photothermal therapy and
in material sciences, where the question of the temperature reached by the
particles has been open for more than $20$ years. Testing the method in real
situations is the next logical step but was outside the time frame of the
thesis.

With the experience built on the anti-Stokes luminescence, characterizing the
scattering of single gold nanoparticles at different temperatures did not prove
to be particularly challenging. Since the first inception of the computer
software until the last version, that allowed to acquire all the data in Chapter
\ref{ch:Damping} almost $4$ years had passed. 

This work summarizes a lot of effort by a lot of people. It neglects all the
failed experiments and frustrations. It is important to remind that failure is
only a relative measure; while we learn something either of nature or of
ourselves, we are being successful. 

\begin{flushright}
{\makeatletter\itshape
    \@firstname\ \@lastname \\
    Leiden, March 2017
\makeatother}
\end{flushright}

