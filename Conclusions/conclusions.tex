\chapter{Conclusions \& Outlook}
\label{conclusions}

\section{General conclusions}
\dropcap{T}{his} thesis is a collection of heterogeneous results that range from
etching of single gold nanorods to studying their anti-Stokes luminescence. Gold
nanoparticles have been in the spotlight for almost two decades because of their
optical properties\cite{Zijlstra2011}. They are ideal candidates for
labelling\cite{Leduc2013} applications and also as
biosensors\cite{Zijlstra2012}. Many properties of the nanoparticles have been
already characterized but there is still a large number of them that needs to be
addressed; this thesis provides several illustrations of this need.

Wet chemical synthesis of nanorods yields a high degree of heterogeneity between
individual particles\cite{Lee2013}. This was already observed in our group when
measuring the quantum yield (QY) of particles with different aspect
ratios\cite{Yorulmaz2012}. The values differ by almost one order of magnitude
between particles that, up to experimental accuracy, should have been identical.
In every chapter of this thesis single-particle results have always been
complemented with statistics.

Chapter \ref{ch:KCN} shows that the mean behavior of single particles is
different from what is observed in bulk suspension. Chapters \ref{ch:Imaging}
and \ref{ch:AntiStokes} focus on the anti-Stokes luminescence, a phenomenon
greatly overlooked in the past decade. Chapter \ref{ch:Damping} on plasmon width
is again an example of the heterogeneity observed at single-particle level.
Experiments similar to these need proper statistics to be complete.

The four chapters of this thesis are but a proof that there is still room for
investigation at single-particle level. Many intriguing phenomena can still be
left to discover. 

\section{Outlook}
Every chapter includes a conclusion regarding the content of the chapter itself.
This section on the other hand aims at pointing out what are the different
possibilities that every chapter opens for future research.

\subsection{Cyanide etching}
Chapter \ref{ch:KCN} shows that is possible to change the shape of gold
nanoparticles once immobilized on a glass coverslip. We employed cyanide etching
because of its well understood chemistry with gold but the methodology is not
limited to it; other reactions are possible alternatives. Moreover we have shown
that it is possible to monitor the changes of shape by studying the evolution of
the plasmon resonance and therefore the experiments can be performed under an
optical microscope. 

Other works have focused into the possibility of using the plasmon resonance
shift as a detector of minute concentrations of cyanide\cite{Wei2012}.
At a single particle level we can easily detect $\uM$ concentrations and $\nM$
should be reachable without changes to the setup. Lowering the concentrations
keeping reasonable measurement times reduces to improving the detection of the
plasmon shift. However gold nanoparticles are completely etched away after being
exposed to cyanide ions for enough time. This would make samples non reusable.

Another interesting opportunity is the ability to change the spacing between
particles with sub nanometer accuracy\cite{Funston2009}. Gold nanorods are
becoming promising nano antennas, and dimers of particles have a much stronger
near field. However, controlling the spacing between particles is a major
challenge. The results shown in chapter \ref{ch:KCN} can be extended to dimers,
where slow etching of the surface of the particles can be used for tuning the
distance between them.

\subsection{Background suppression}
Chapter \ref{ch:Imaging} shows that it is possible to image gold nanorods in
high background conditions by detecting their anti-Stokes emission. The chapter
focuses into imaging under living cells but the technique is not limited to
biological applications. High background conditions can include working with
fluorescent molecules in solution, for example to study enhanced
FCS\cite{Langguth2014}. Background suppression is not the only advantage of
anti-Stokes imaging.

A common problem in colocalization studies is the correction for chromatic
aberrations and misalignment of different beams. If one desires to colocalize a
gold nanorod and a fluorescent dye with absorption in the same spectral region,
the anti-Stokes emission provides a way to achieve it with only one excitation
laser and one detection path. Employing a single laser beam rules out the
possibility of a misalignment of the excitation path; the detection of both
channels (anti-Stokes for the rods and Stokes for the dye) can be concentrated
over a short spectral range, thus minimizing chromatic aberrations.

Colocalizing gold nanorods and fluorescently labelled proteins can give insight
into the different processes that mediate the uptake of gold
nanorods\cite{Leduc2013}. It can also be useful for characterizing the targeting
of proteins in living cells. A gold nanorod can be functionalized to bind to
specific proteins\cite{Li2013a}; the binding efficiency and specificity,
however, are difficult to determine \textit{in vivo} if there is a high background
signal.

Anti-Stokes detection is not limited to imaging. If used as labels, nanorods can
be used for tracking\cite{Spillane2014} specific proteins for extended periods
of time. In living cells, regulatory mechanisms depend on free diffusion and
active transport\cite{Nowack2012}. Tracking of functionalized single
nanoparticles can provide important insight into mechanisms that are active over
different timescales\cite{Conde2013}.

\subsection{Temperature sensing with anti-Stokes luminescence}
Chapter \ref{ch:AntiStokes} shows that anti-Stokes luminescence from single
nanoparticles can be used for nano-thermometry. This novel result opens many
possibilities in the fields of photothermal therapy\cite{Hirsch2003} and nano
fabrication\cite{Fedoruk2013}. For over $20$ years gold nanoparticles have been
studied as possible candidates for treating cancer\cite{ONeal2004}. A large
community is focused into using nanoparticles to locally increase the temperature of malignant cells,
preserving the healthy ones.

After decades of research, however, there is little insight into the temperature
that the nanoparticles have to reach to induce cellular death\cite{Huang2006a}.
The conclusions of chapter \ref{ch:AntiStokes} clearly show that the technique
developed is ready to be implemented in biologically relevant conditions. Being
able to actively control and monitor the temperature of nanoparticles in or
around cells has never been done before and can yield important answers to the
mechanisms that induce cell death.

Moreover the method described in chapter \ref{ch:AntiStokes} can be used to
measure the temperature of nanoparticles in various situations. For instance the
characterization of optically trapped nanoparticles normally relies on
assumptions of the temperature\cite{Ruijgrok2011a}. Nano bubble
generation\cite{Hou2015}, polymerization at the nanoscale\cite{Ma2014a},
controlled chemical reactions\cite{Urban2009}, photothermal
detection\cite{Boyer2002} are some of the fields where actually measuring the
temperature of the nanoparticles instead of estimating it can provide insight
into new phenomena.

An important task for future work should be to characterize different particle
geometries. Gold shells\cite{Gobin2007}, bipyramids\cite{Rao2015}, even spheres
of different diameters can be better suited for temperature sensing. Different
plasmon resonances and different quantum yields can make other particles better
anti-Stokes emitters.

Acquiring spectra as was done in chapter \ref{ch:AntiStokes} is a slow process;
it can take several minutes to obtain a proper signal-to-noise ratio. There is a
possibility to shorten the acquisition times by studying the ratio of
anti-Stokes to Stokes emission\cite{Pozzi2015}. In principle the Stokes emission
is constant with temperature and depends only on the laser power; the
anti-Stokes however will be brighter for higher temperatures. Already in figure
\ref{fig:ASS-ratio} it is possible to observe that the ratio of both types of
emission can be easily reproduced by numerical calculations. Preliminary
calculations show that the ratio of anti-Stokes to Stokes changes with
temperatures, but experimental data is missing.

\subsection{Plasmon Damping}
Chapter \ref{ch:Damping} shows the relation between the plasmon damping rate and
the temperature of the medium surrounding the nanoparticles. The main idea of
the chapter was to explore the possibility of using the broadening of the
plasmon resonance as an alternative thermometry strategy. Anti-Stokes
luminescence has the advantage of not needing a calibration, but the higher
laser powers employed induce a temperature rise that can be much higher than the
temperature to be detected.

Chapter \ref{ch:Damping} shows that, on average, the linear relationship between
the plasmon full width at half maximum and temperature agrees with the expected
value from bulk gold. However there is a big heterogeneity between 
nanoparticles, not all of them have the same broadening rate. This explains why
the broadening was not observed in bulk suspension and also sets a limit to the
applicability of this method for temperature measurements. 

Since every nanoparticle behaves in slightly a different way when increasing
temperature, one needs to build proper statistics to determine how much the
temperature of the sample increased or decreased. The statistics can be built
either by studying several individual nanoparticles, as was done in this thesis,
or by placing a bigger number of particles in the focal spot, as done with
quantum dots\cite{Li2007}. The heterogeneity, however, can pose a limit to the
applicability of the method. It is possible that other geometries such as
bipyramids exhibit a more homogeneous behavior.

\references{dissertation}