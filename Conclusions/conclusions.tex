\chapter{Conclusions \& Outlook}
\label{conclusions}

\section{General conclusions}
\dropcap{T}{his} thesis is a collection of heterogeneous results that range from
etching of single gold nanorods to studying their anti-Stokes luminescence. Gold
nanoparticles have been in the spotlight for almost two decades because of their
optical properties. They are ideal candidates in labelling applications but also
as biosensors. Many properties of the nanoparticles have been already
characterized but there is still a large number of them that needs to be
addressed; this thesis is an example of this.

Wet chemical synthesis of nanorods yields a high degree of heterogeneity between
individual particles. This was already observed in our group when measuring the
quantum yield (QY) of particles with different aspect ratios. The values differ
in almost one order of magnitude between particles that up to experimental
accuracy should have been identical. Every chapter of this thesis shows that
single particle results always have to be complemented with proper statistics.

Chapter \ref{ch:KCN} shows that the mean behavior of single particles is
different than what observed in bulk suspension. Chapters \ref{ch:Imaging} and
\ref{ch:AntiStokes} focus on the anti-Stokes luminescence, a phenomenon greatly
overlooked in the past decades. Chapter \ref{ch:Damping} is again an example of
the heterogeneity observed at a single particle level. Experiments similar to
those need proper statistics to be complete. 

The four chapters of this thesis are but a proof that there is still room for
investigation at a single particle level. Many intriguing phenomena can still be
left to discover. Moreover it is time to start applying the acquired knowledge
at a single particle level as building blocks for other disciplines. 

\section{Outlook}
In every chapter consl

\subsection{Cyanide etching}
Chapter \ref{ch:KCN} shows that is possible to change the shape of gold
nanoparticles once immobilized on a glass coverslip. We employed cyanide etching
because of its well understood chemistry with gold but the methodology is not
limited to it; other reactions are possible alternatives. Moreover we have shown
that it is possible to monitor the change of shape by studying the evolution of
the plasmon resonance.

Some other works focused into the possibility of using the plasmon resonance
shift as a detector of minute concentrations of cyanide. At a single particle
level we can easily detect $\uM$ concentrations and $\nM$ should be reachable
without changes to the setup. Lowering the concentrations keeping reasonable
measurement times reduces to improving the detection of the plasmon shift.
However gold nanoparticles are completely etched away after being exposed to
cyanide ions for enough time. This would make samples non reusable.

Another interesting opportunity is the ability to change the spacing between
particles with sub nanometer accuracy. Gold nanorods are becoming promising nano
antennas, and dimers of particles have much stronger near fields. However
controlling the spacing between particles is a major challenge. The results
shown in chapter \ref{ch:KCN} can be extended to dimers aiming to the
fabrication of nano antennas with variable inter particle distances.

\subsection{Background suppression}
Chapter \ref{ch:Imaging} shows that it is possible to image gold nanorods in
high background conditions by detecting their anti-Stokes emission. The chapter
focuses into imaging under living cells but the technique is not limited to
biological applications. High background conditions can include working with
fluorescent molecules in solution, for example to study enhanced FCS. Background
suppression is not the only advantage of anti-Stokes imaging.

A common problem in colocalization studies is the correction for chromatic
aberrations and misalignment of different beams. If one desires to colocalize a
gold nanorod and a fluorescent dye with absorption in the same spectral region,
the anti-Stokes emission provides a way to achieve it with only one excitation
laser. Employing a single laser beam discards the possibility of a misalignment
of the excitation path; the detection of both channels (anti-Stokes for the rods
and Stokes for the dye) can span over a short spectral range, thus
minimizing chromatic aberrations.

Colocalizing gold nanorods and fluorescently label proteins can give insight
into the different processes that mediate the uptake of gold nanorods. It can
also be useful for characterizing the targeting of proteins in living cells. A
gold nanorod can be functionalized to bind to specific proteins; the binding
efficiency and specificity however are difficult to determine \textit{in vivo}.

\subsection{Temperature sensing with anti-Stokes luminescence}
Chapter \ref{ch:AntiStokes} shows that anti-Stokes luminescence from single
nanoparticles can be used for nano thermometry. This novel result opens many
possibilities in the fields of photothermal therapy and nano fabrication. For
over $20$ years gold nanoparticles have been studied as possible candidates for
treating cancer. A large community is focused into using nanoparticles to
locally increase the temperature of malignant cells, preserving the healthy
ones.

After decades of research, however, there is little insight on the temperature
that the nanoparticles have to reach to induce cellular death. The conclusions
of chapter \ref{ch:AntiStokes} clearly show that the technique developed is
ready to be implemented in biologically relevant questions. Being able to
actively control and monitor the temperature of nanoparticles in or around cells
has never been done before and can yield important answers to the mechanisms
that induce cell death. 

Moreover the method described in chapter \ref{ch:AntiStokes} can be used to
measure the temperature of nanoparticles in various situations. For instance
the characterization of optically trapped nanoparticles normally relies on
assumptions of the temperature. Nano bubble generation, polimerization at the
nanoscale, controlled chemical reactions, photothermal detection are some of the
fields where actually measuring the temperature of the nanoparticles instead
of estimating it can provide insight into new phenomena. 

\subsection{Plasmon Damping}
Chapter \ref{ch:Damping}, the last section of this thesis, shows the relation
between the plasmon damping rate and the temperature of the medium surrounding
the nanoparticles. The main idea of the chapter was to explore the possibility
of using the broadening of the plasmon resonance as an alternative thermometry
strategy. Anti-Stokes luminescence has the advantage of not needing a
calibration, but the higher laser powers employed induce a temperature rise that
can be higher than the temperature to be detected.

Chapter \ref{ch:Damping} shows that, on average, the linear relationship between
the plasmon full width at half maximum and temperature agree with the expected
value from bulk gold. However there is a big heterogeneity between 
nanoparticles, not all have the same broadening rate. This explains why the
broadening was not observed in bulk suspension and also sets a limit to the
applicability for temperature measurements. 

Since every nanoparticle behaves in slightly a different way when increasing
temperature, on needs to build proper statistics to determine how much the
temperature of the sample increased or decreased. 


\section{Building on software}

