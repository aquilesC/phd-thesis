\chapter*{Summary}
\addcontentsline{toc}{chapter}{Summary}
\setheader{Summary}

Gold nanorods are ideal candidates for complementing fluorophores in labelling
applications. The presence of the surface plasmon resonance generates large
absorption and scattering cross sections, thus making the detection of single
nanoparticles possible under a light microscope. The plasmon of gold nanorods
depends on the ratio between their width and length and covers the range between
$540\nm$ for spheres and even above $800\nm$ for elongated particles, thus
almost the entire visible and near-infrared spectrum. The surface plasmon
presents great opportunities in (bio-)sensing, enhanced spectroscopies,
photothermal therapy and for concentrating light below the diffraction limit.

Chapter 1 of this thesis is a brief overview on fluorescence microscopy and on
the basic properties of gold nanoparticles. Microscopy and specifically
fluorescence microscopy is the result of a long process of technical
improvements in optics and light sources, but also of the labels used to prepare
the samples. From simpler molecules to genetically encoded proteins, the wealth
of resources available nowadays is remarkable. In this context, gold
nanoparticles can find their way because of their stability over time. 

The resonance wavelength (or energy) of metallic nanoparticles will be given by
their geometry and by the surrounding medium's properties, such as its
refractive index. The geometry of the particles is determined during the
synthesis procedure, where the average length and width can be tuned. Once the
particles are deposited on a substrate, their resonance is already determined.
It is possible, however, to induce shape modification to the particles through
chemical means.

Previous works have focused on bulk measurements in suspension. In this case,
the tips of the particles tend to be more reactive because they are less
protected by the surfactants that prevent aggregation of particles. This leads
to an anisotropic reaction that slowly transforms elongated particles into
spheres and that softens sharp edges or tips, yielding an overall blue-shift of
the resonance.

Chapter 2 shows that through well known chemistry between gold and cyanide ions
it is possible to induce a red-shift of the plasmon. This is modelled through an
isotropic etching of the particles, and a good agreement between calculations
and experiments is obtained. The main difference with previous work is the
absence of a capping agent on the particles' surface. Controllably changing the
shape of nanoparticles is of great importance for experiments where a specific
resonance is needed.

When particles are excited by a monochromatic light source, such as a laser,
they will emit light at different wavelengths than the excitation wavelength.
This emission is generally referred to as luminescence and is commonly used for
imaging and tracking nanoparticles under confocal microscopes. When the
excitation wavelength coincides with the plasmon resonance, the emission will
happen not only at longer wavelengths, i.e. lower energies, but also at shorter
wavelengths. This emission is called anti-Stokes emission and possesses
intriguing properties.

Chapter 3 shows that it is possible to image gold nanorods in biologically
relevant conditions through detection of their anti-Stokes emission. By placing
a short-pass filter in the detection path the background level is reduced
significantly, while the luminescence signal from the particles remains high.
This is valid even for cells stained with a dye with high quantum yield
that absorbs light of the same wavelengths as the rods. In these conditions it
is not possible to observe any single nanoparticle through conventional
Stokes-shifted emission while the anti-Stokes scheme presents a
signal-to-background ratio higher than $10$.

The technique presented in chapter 3 can be readily implemented in any
conventional microscope by the addition of the appropriate filters. It does not
require any special operation nor infrastructure. Moreover any data analysis
tool for tracking, imaging, centroid extraction, etc. of single labels can
readily be implemented without further modifications. The results of chapter 3
can have a major impact in the way nanoparticles are imaged and detected in
biological conditions.

During the past two decades there has been an increasing interest in gold
nanoparticles as possible agents for medical treatments. The strong interaction
between particles and light makes them ideal candidates not only for labelling
but also for releasing heat into very localized environments. This simple
approach can be used, for instance, to induce death of cancer cells and is
normally referred to as plasmonic photo thermal therapy.

Chapter 4 focuses on the characterization of the mechanisms that give rise to
anti-Stokes luminescence. Discarding multi-photon processes, photons with higher
energies than the excitation energy require interactions with thermal baths. In
a nanoparticle electrons and holes can interact with phonons before recombining
radiatively. The distribution of phonons in gold follows Bose-Einstein
statistics, where the only free parameter is the temperature. We
therefore propose in the chapter that anti-Stokes emission can be used for
sensing temperature at the nanoscale.

By carefully fitting the luminescence spectra of single gold nanorods and
nanospheres it is possible to extract the surface temperature of the particles.
The method presented in chapter 4 does not depend on any ad-hoc calibration and
can be performed in any confocal microscope with a coupled spectrometer. The
chapter shows the increase in temperature with increasing laser powers and also
shows the changes that the luminescence spectra undergo when increasing the
medium temperature. The calibration-free procedure is a major improvement over previous techniques
in the field of nano-thermometry.

Luminescence is not the only method for detecting gold nanorods with an optical
microscope. Gold nanoparticles have a large scattering cross section coinciding
with the plasmon resonance. Exciting nanoparticles with white light allows one
to record the scattering spectra in any confocal microscope coupled to a
spectrometer. Since the plasmon damping rate is affected by the surrounding
conditions it can also be used to detect changes in temperature. From the
mechanisms involved to explain the plasmon damping rate, only electron-phonon
coupling is dependent on temperature. 

Chapter 5 focuses on the characterization of the plasmon resonance of single
gold nanorods at various temperatures. In the range of temperatures studied
(between $293\K$ and $350\K$), the plasmon width increases linearly with
temperature. The broadening is assigned to an increase in the electron-phonon
damping rate. Measuring the broadening of the resonance can then be related to
changes in temperature of the surrounding medium. The powers needed for
recording scattering spectra are much lower than the ones employed when exciting
the luminescence of the particles. However the broad distribution of widths and
broadening rates found in the studies of chapter 5 does not allow
to perform an absolute temperature measurement but only to measure a relative
change.


\chapter*{Samenvatting}
\addcontentsline{toc}{chapter}{Samenvatting}
\setheader{Samenvatting}

{\selectlanguage{dutch} Gouden nanostaafjes zijn ideale kandidaten om
fluoroforen te complementeren als labels. De \textit{surface plasmon resonance}
zorgt voor een grote werkzame doorsnede voor absorptie en verstrooiing, waardoor zelfs
enkelvoudige nanodeeltjes onder een lichtmicroscoop zichtbaar zijn. Het plasmon
van gouden nanostaafjes hangt af van de verhouding tussen hun lengte en breedte,
en beslaat de golflengten tussen $540\nm$ voor sferische deeltjes, tot meer dan
$800\nm$ voor langwerpige staafjes, oftewel bijne het gehele zichtbare en
nabij-infrarode spectrum. Het oppervlakteplasmon biedt grote mogelijkheden voor
\textit{(bio-)sensing}, geamplificeerde spectroscopie, fotothermische therapie
en voor het concenteren van licht voorbij de diffractielimiet. 

Hoofdstuk 1 van deze thesis biedt een kort overzicht van
fluorescentiemicroscopie en de basiseigenschappen van gouden nanodeeltjes.
Microscopie en specifiek fluorescentiemicroscopie zijn resultaten van een lange
technische ontwikkeling van de optica en van lichtbronnen, maar ook van de
labels die worden gebruikt om de monsters te prepareren. Het scala aan
tegenwoordig beschikbare middelen is opmerkelijk: van simpele moleculen tot
genetisch ge\"{e}ncodeerde eiwitten. Gouden nanodeeltjes verdienen hun plek
tussen deze gereedschappen vanwege hun lange stabiliteit. 

De resonantiefrequentie (of energie) van metallische nanodeeltjes wordt bepaald
door hun geometrische eigenschappen en door de eigenschappen, zoals de
brekingsindex, van het medium waarin ze zich bevinden. De geometrische
eigenschappen worden bepaald tijden het synthetiseren van de deeltjes, wanneer
de gemiddelde lengte en breedte kunnen worden afgestemd. Wanneer de deeltjes op
een substraat worden neergelaten, ligt hun al resonantie vast. Echter, het is
mogelijk om de vorm aan te passen via chemische methoden. 

Voorgaand werk heeft zich geconcentreerd op bulkmetingen in suspensie. In dit
geval zijn de uiteinden van de deeltjes vaak meer reactief, omdat ze minder goed
worden beschermd door de surfactanten die het aggregeren van deeltjes voorkomen.
Dit leidt tot een anisotropische reactie die langzaam de langwerpige deeltjes
tot sferische deeltjes zal reduceren, en die scherpe randen en punten zal
eroderen, hetgeen leidt tot een blauwverschuiving van de resonantie. 

Hoofdstuk 2 toont aan dat het ook mogelijk is, via welbekende chemische reacties
tussen goud en cyanide-ionen, om een roodverschuiving teweeg te brengen. Dit
proces kan worden gemodelleerd als het isotropisch etsen van de deeltjes, en we
vinden een goede overeenkomst tussen onze berekeningen en experimentele
waarnemingen. Het voornaamste verschil met voorgaand werk is de afwezigheid van
een surfactant op het oppervlak van de deeltjes. Het beheerst aanpassen van de
vorm van nanodeeltjes is van groot belang voor experimenten waar een specifieke
resonantie vereist is.

Wanneer deeltjes worden aangeslagen door een monochromatische lichtbron, zoals
een laser, zenden ze in het algemeen licht uit bij een andere golflengte dan die
van de bron. Dit noemen we luminescentie en dit wordt veel gebruikt voor het
afbeelden en volgen van nanodeeltjes onder confocale microscopen. Wanneer de
excitatiegolflengte gelijk is aan de resonantiefrequentie van het plasmon, vindt
de emissie niet alleen plaats bij langere golflengten, i.e. lagere
energie\"{e}n, maar ook bij kortere golflengten. Dit noemen we
anti-Stokes-emissie en dit proces kent intrigerende eigenschappen.

Hoofdstuk 3 toont aan dat het mogelijk is om gouden nanostaafjes te detecteren
in biologisch relevante omstandigheden via hun anti-Stokes-emissie. Een
short-pass filter in het detectiepad resulteert in een significante
onderdrukking van de achtergrondruis, terwijl het luminescentiesignaal hoog
blijft. Dit geldt zelfs voor cellen gekleurd met een dye met een hoge
\textit{quantum yield} die bij dezelfde golflengte licht absorbeert als de
staafjes. Onder dergelijke omstandigheden is het niet mogelijk om losse
nanodeeltjes zichtbaar te maken via conventionele Stokes-verschoven emissie,
terwijl de anti-Stokesmethode een signaal-ruisverhouding oplevert van meer dan
10.

De techniek gepresenteerd in Hoofdstuk 3 kan eenvoudig worden
ge\"{i}mplementeerd in elke conventionele microscoop door de juiste filters toe
te voegen. Er zijn geen speciale operaties of infrastructuur voor nodig. Bovendien kunnen analyse-tools
voor het volgen, afbeelden, extraheren van zwaartepunten, etc. van enkelvoudige
labels worden gebruikt zonder aanpassingen. De resultaten van Hoofdstuk 3 kunnen
een grote impact hebben op de manier waarop nanodeeltjes worden gedetecteerd en
afgebeeld in biologische omstandigheden. 

Gedurende de laatste twee decennia is er een groeiende interesse geweest in
gouden nanodeeltjes als mogelijke middelen in medische behandelingen. De sterke
interactie tussen de deeltjes en licht maakt ze ideaal geschikt niet alleen als
labels, maar ook om zeer gelokaliseerd warmte los te laten. Deze simpele
toepassing kan bijvoorbeeld worden gebruikt om de dood van kankercellen te
induceren, en wordt over het algemeen plasmonische fotothermische therapie
genoemd.

Hoofdstuk 4 is gericht op het karakteriseren van de mechanismen die
anti-Stokes luminescentie veroorzaken. Behalve in processen waar
meerdere fotonen mee gemoeid zijn, moet het systeem in contact staan met een thermisch reservoir
om fotonen met een hogere energie dan de excitatie-fotonen uit te kunnen zenden.
In een nanodeeltjes kunnen electronen en gaten ook met fononen interacteren,
voordat ze recombineren en licht uitzenden. De distributie van fononen in goud
volgt de Bose-Einsteinstatistiek, waarbij de temperatuur de enige vrije
parameter is. We stellen daarom in Hoofdstuk 4 voor dat anti-Stokes-emissie ook
kan worden gebruikt om op de nanoschaal temperaturen te meten. 

Door zorgvuldig de luminescentiespectra van enkelvoudige gouden nanostaafjes en
nanosferen te meten is het mogelijk om de oppervlaktetemperatuur van de deeltjes
te bepalen. Het ad-hoc kalibreren van de methode uit Hoofdstuk 4 is niet nodig,
en de methode kan worden uitgevoerd met elke confocale microscoop met een
gekoppelde spectrometer. In dit hoofdstuk laten we zien hoe de temperatuur
toeneemt met toenemend laservermogen, en de veranderingen die de
luminescentiespectra ondergaan wanneer we de temperatuur van het medium
verhogen.

Deze kalibratievrije methode is een grote verbetering ten op zichte van
voorgaande technieken in de nanothermometrie. De resultaten uit dit hoofdstuk
kunnen een significante impact hebben op een opkomende gemeenschap die een van
de meest dringende gezondheidskwesties van onze tijd probeert aan te pakken.

Luminescentie is niet de enige manier om gouden nanostaafjes te detecteren met
een optische microscoop. Gouden nanodeeltjes hebben een grote
verstrooiingsdoorsnede die samenvalt met de plasmonresonantie. Wanneer we
nanodeeltjes aanslaan met wit licht kunnen we de verstrooiingsspectra meten in
elke confocale microscoop met een gekoppelde spectrometer. Omdat de demping van
het plasmon afhangt van de eigenschappen van het medium, kan ook dit worden
aangewend om veranderingen in temperatuur te meten. Van de mechanismen die een
rol spelen in deze demping hang alleen de koppeling tussen electronen en fononen
af van de temperatuur.

Hoofdstuk 5 is gericht op het karakteriseren van de plasmonresonantie van
enkelvoudige gouden nanostaafjes bij verschillende temperaturen. Bij de
temperaturen die we hebben bestudeerd (tussen $293\K$ en $350\K$) neemt de
breedte van de resonantiepiek lineair toe met de temperatuur. Deze verbreding kennen we
toe aan een versterkte electron-fonondemping. Het meten van deze verbreding kan
daarom worden gerelateerd aan temperatuurveranderingen in het omringende medium.
Het meten van verstrooiingsspectra kost veel minder vermogen dan het exciteren
van de lumeniscentie van de deeltjes. Echter, in de studies van Hoofdstuk 5
vinden we dusdanig brede verdelingen van piekbreedtes, dat het meten van de
absolute temperatuur niet mogelijk is, en we deze methode slechts kunnen
aanwenden om veranderingen in temperatuur te meten. }
