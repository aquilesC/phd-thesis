\chapter*{Summary}
\addcontentsline{toc}{chapter}{Summary}
\setheader{Summary}

Gold nanorods are ideal candidates for complementing fluorophores in labelling
applications. The presence of the surface plasmon resonance generates large
absorption and scattering cross sections, thus making the detection of single
nanoparticles possible under a light microscope. The plasmon of gold nanorods
depends on the ratio between their width and length and covers the range between
$540\nm$ for spheres and even above $800\nm$ for elongated particles, thus
almost the entire visible and near-infrared spectrum. The surface plasmon
presents great opportunities in (bio-)sensing, enhanced spectroscopies,
photothermal therapy and for concentrating light below the diffraction limit.

Chapter 1 of this thesis is a brief overview on fluorescence microscopy and on
the basic properties of gold nanoparticles. Microscopy and specifically
fluorescence microscopy is the result of a long processes of technical
improvements in optics and light sources, but also of the labels used to prepare
the samples. From simpler molecules to genetically encoded proteins, the wealth
of resources available nowadays is remarkable. In this context, gold
nanoparticles can find their way because of their stability over time. 

The resonance wavelength (or energy) of metallic nanoparticles will be given by
their geometry and by the surrounding medium's properties, such as its
refractive index. The geometry of the particles is determined during the
synthesis procedure, where the average length and width can be tuned. Once the
particles are deposited on a substrate, their resonance is already determined.
It is possible, however, to induce shape modification to the particles through
chemical means.

Previous works have focused on bulk measurements in suspension. In this case,
the tips of the particles tend to be more reactive because they are less
protected by the surfactants that prevent aggregation of particles. This leads
to an anisotropic reaction that slowly transforms elongated particles into
spheres and that softens sharp edges or tips, yielding an overall blue-shift of
the resonance.

Chapter 2 shows that through well known chemistry between gold and cyanide ions
it is possible to induce a red-shift of the plasmon. This is modelled through an
isotropic etching of the particles, and a good agreement between calculations
and experiments is obtained. The main difference with previous work is the
absence of a capping agent on the particles' surface. Controllably changing the
shape of nanoparticles is of great importance for experiments where a specific
resonance is needed.

When particles are excited by a monochromatic light source, such as a laser,
they will emit light at different wavelengths than the excitation wavelength.
This emission is generally referred to as luminescence and is commonly used for
imaging and tracking nanoparticles under confocal microscopes. When the
excitation wavelength coincides with the plasmon resonance, the emission will
happen not only at longer wavelengths, i.e. lower energies, but also at shorter
wavelengths. This emission is called anti-Stokes emission and possesses
intriguing properties.

Chapter 3 shows that it is possible to image gold nanorods in biologically
relevant conditions through detection of their anti-Stokes emission. By placing
a short-pass filter in the detection path the background level is reduced
significantly, while the luminescence signal from the particles remains high.
This is valid even for cells stained with a dye with high quantum yield
that absorbs light of the same wavelengths as the rods. In these conditions it
is not possible to observe any single nanoparticle through conventional
Stokes-shifted emission while the anti-Stokes scheme presents a
signal-to-background ratio higher than $10$.

The technique presented in chapter 3 can be readily implemented in any
conventional microscope by the addition of the appropriate filters. It does not
require any special operation nor infrastructure. Moreover any data analysis
tool for tracking, imaging, centroid extraction, etc. of single labels can
readily be implemented without further modifications. The results of chapter 3
can have a major impact in the way nanoparticles are imaged and detected in
biological conditions.

During the past two decades there has been an increasing interest in gold
nanoparticles as possible agents for medical treatments. The strong interaction
between particles and light makes them ideal candidates not only for labelling
but also for releasing heat into very localized environments. This simple
approach can be used, for instance, to induce death of cancer cells and is
normally referred to as plasmonic photo thermal therapy.

Chapter 4 focuses into the characterization of the mechanisms that give rise to
anti-Stokes luminescence. Discarding multi-photon processes, photons with higher
energies than the excitation energy require interactions with thermal baths. In
a nanoparticle electron and holes can interact with phonons before recombining
radiatively. The distribution of phonons in gold follows Bose-Einstein
statistics, where the only free parameter is the temperature. We
therefore propose in the chapter that anti-Stokes emission can be used for
sensing temperature at the nanoscale.

By carefully fitting the luminescence spectra of single gold nanorods and
nanospheres it is possible to extract the surface temperature of the particles.
The method presented in chapter 4 does not depend on any ad-hoc calibration and
can be performed in any confocal microscope with a coupled spectrometer. The
chapter shows the increase in temperature with increasing laser powers and also
shows the changes that the luminescence spectra undergo when increasing the
medium temperature.

The calibration-free procedure is a major improvement over previous techniques
in the field of nano-thermometry. The results from the chapter can have a
significant impact on an emerging community that addresses one of the most
pressing health issues of our time.

Luminescence is not the only method for detecting gold nanorods with an optical
microscope. Gold nanoparticles have a large scattering cross section coinciding
with the plasmon resonance. Exciting nanoparticles with white light allows one
to record the scattering spectra in any confocal microscope coupled to a
spectrometer. Since the plasmon damping rate is affected by the surrounding
conditions it can also be used to detect changes in temperature. From the
mechanisms involved to explain the plasmon damping rate, only electron-phonon
coupling is dependent on temperature. 

Chapter 5 focuses on the characterization of the plasmon resonance of single
gold nanorods at various temperatures. In the range of temperatures studied
(between $293\K$ and $350\K$), the plasmon width increases linearly with
temperature. The broadening is assigned to an increase in the electron-phonon
damping rate. Measuring the broadening of the resonance can then be related to
changes in temperature of the surrounding medium. The powers needed for
recording scattering spectra are much lower than the ones employed when exciting
the luminescence of the particles. However the broad distribution of widths and
broadening rates found in the studies of chapter 5 does not allow
to perform an absolute temperature measurement but only to measure a relative
changes


\chapter*{Samenvatting}
\addcontentsline{toc}{chapter}{Samenvatting}
\setheader{Samenvatting}

{\selectlanguage{dutch}

Samenvatting in het Nederlands\ldots

}

